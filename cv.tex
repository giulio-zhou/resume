%\let\nofiles\relax
\documentclass{article}
\usepackage[top=0.7in,  bottom=0.4in, left = 0.75in, right=0.75in]{geometry}
\usepackage{anyfontsize}
%\makeatletter
%\def\@tablebox#1{\begin{tabular}[t]{@{}c@{}}#1\end{tabular}}
%\makeatother

%\addtolength{\voffset}{-1.5cm}
%\addtolength{\textheight}{1cm}
%\addtolength{\hoffset}{-0.5cm}
%\addtolength{\textwidth}{1cm}
\pdfpagewidth 8.5in
\pdfpageheight 11in
\begin{document}
\newcommand{\HRule}{\rule{\linewidth}{0.2mm}}

%\name{\LARGE{Giulio Zhou}\\}
%\address{1655 Euclid Avenue, Berkeley, CA 94709\\
%giuliozhou8@gmail.com\\
%(925) 997-8192\\}
%\begin{center}
%\noindent
%\textbf{{\LARGE Giulio Zhou}}\vspace{-3.4mm}
%1655 Euclid Avenue, Berkeley, CA 94709\\
%\end{center}
%\fontsize{10.7}{14.4}
%\selectfont

\begin{center}
\textbf{{\LARGE Giulio Zhou}} \\ 
% \textbf{\fontsize{11}{13.2} Email:} giuliozhou8@gmail.com  $|$ \textbf{Website:} giuliozhou.com\\
% \textbf{LinkedIn}: linkedin.com/in/giuliozhou $|$ \textbf{GitHub}: github.com/giulio-zhou\\[2mm]
\textbf{\fontsize{11}{13.2} Email:} giuliozhou8@gmail.com  $|$ \textbf{Website:} giuliozhou.com $|$ \textbf{GitHub}: github.com/giulio-zhou\\[2mm]
\end{center}

%\begin{resume}
%\moveleft\hoffset\vbox{\hrule width\resumewidth height 0.8pt}\smallskip
% \begin{figure}[!htb]
% \minipage{0.35\textwidth}
% \fontsize{10.0}{14.4}
% \selectfont
% \textbf{LinkedIn}: linkedin.com/in/giuliozhou\\[-1.4mm]
% \textbf{Email:} giuliozhou8@gmail.com
% \endminipage
% \minipage{0.3\textwidth}
% \fontsize{22.0}{17.2}
% \selectfont
% \center
% \textbf{{Giulio Zhou}}
% \endminipage
% \minipage{0.35\textwidth}
% \fontsize{10.0}{14.4}
% \selectfont
% \raggedleft
% \textbf{GitHub}: github.com/giulio-zhou\\[-1.4mm]
% \textbf{Phone:} (925) 997-8192
% \endminipage
% \end{figure}

\normalsize
\noindent
\textbf{{\Large Education}}\\[-2mm]
\HRule\\
\textbf{University of California, Berkeley (Berkeley, CA)}
\hfill 08/2012 -- 05/2016 \\
\indent
\textbf{Bachelor of Arts, Computer Science}  
\hfill{\textbf{Cumulative GPA: }3.891}\\
\textit{Relevant Coursework}: Machine Learning, Artificial Intelligence, Computer Vision, Operating Systems, Image Processing, Probability and Random Processes, Data Structures, Computer Networking, Database Systems.
\\

\noindent
\textbf{{\Large Research Experience}}\\[-2mm]
\HRule\\
\noindent
\textbf{Algorithms, Machines and People Lab}
\hfill 05/2016 -- \textit{present}
\begin{itemize}
\vspace{-2.5mm}
\item Worked under Joseph Gonzalez on Clipper, a system for online, low-latency machine learning model serving.
\vspace{-2.5mm}
\item Implemented the REST interface and a C++ RPC server to support Vowpal Wabbit, a scalable library for linear model predictions.
\vspace{-2.5mm}
\item Benchmarked Clipper RPC system on AlexNet, Network-in-Network and Inception Tensorflow models, demonstrating throughput and latency parity with Google's Tensorflow Serving system.
\vspace{-2.5mm}
\item Explored the use of classification and hypothesis testing techniques for real-time covariate shift detection and adaptation (through online reweighted model retraining).
\vspace{-2.5mm}
\item Publication: Daniel Crankshaw, Xin Wang, \textbf{Giulio Zhou}, Joseph E. Gonzalez. \textit{Clipper: A Low-Latency Online Prediction Serving System}. NSDI, 2017. To Appear.
\end{itemize}
\vspace{-2mm}

\noindent
\textbf{Berkeley Artificial Intelligence Research Lab}
\hfill 03/2015 -- \textit{present}
\begin{itemize}
\vspace{-2.5mm}
\item Worked under Stuart Russell on sampling algorithms for Bayesian LOGic (BLOG), an open-universe probabilistic modeling language.
\vspace{-2.5mm}
\item Implemented a Gaussian Mixture Model (with temporal and spatial constraints) for background subtraction in video sequences. Written in 20 lines of BLOG code, the algorithm achieves comparable performance to OpenCV's state-of-the-art background subtraction libraries. Submitted to DARPA as part of DARPA's Probabilistic Programming for Advancing Machine Learning (PPAML) initiative.
\end{itemize}
\vspace{-2mm}

\noindent
\textbf{Nanocrystal Synthesis Lab}
\hfill 01/2014 -- 12/2014
\begin{itemize}
\vspace{-2.5mm}
\item Ran experiments on the NERSC supercomputers, using gradient descent optimization and the generalized moments method to simulate the optical and mechanical properties of tetrapod nanocrystals.
\vspace{-2.5mm}
\item Implemented a 3D lattice-spring model (with Java multithreading) to simulate polymer stress-strain effects.
\end{itemize}

\noindent
\textbf{{\Large Industry Experience}}\\[-2mm]
\HRule\\
\textbf{Google, inc.}
\hfill 05/2015 -- 08/2015\\
\textbf{Software Engineering Intern}
\begin{itemize}
\vspace{-2.5mm}
\item Worked on the Display Ad Automation Team to improve the quality of Native Ads.
\vspace{-2.5mm}
\item Designed and built a backend pipeline for high-quality automated text-to-image matching for internationalized display ads.
\vspace{-2.5mm}
\item Developed quality visualization tools and deployed non-English Native Ads, doubling overall coverage.
\end{itemize}
\vspace{1mm}

\noindent
\textbf{{\Large Teaching Experience}}\\[-2mm]
\HRule\\
\textbf{CS 189/289A, Introduction to Machine Learning (Fall 2016)}
\begin{itemize}
\vspace{-2.5mm}
\item Taught undergraduate and graduate students in two weekly 1-hour discussions, covering topics such as support vector machines, bias-variance tradeoff, classifiers, logistic regression, kernelization, and neural networks.
\end{itemize}
\vspace{-2mm}
\textbf{CS 61BL, Data Structures and Programming Methodology (Summer 2016)}
\begin{itemize}
\vspace{-2.5mm}
\item Prepared daily mini-lectures, developed course material and led 12 hours of laboratory instruction per week.
\end{itemize}
\vspace{-2mm}
\textbf{CS 61B, Data Structures and Algorithms (Spring 2016)}
\begin{itemize}
\vspace{-2.5mm}
\item Held office hours, wrote exam problems and led weekly discussion and laboratory sections.
\vspace{-2.5mm}
\item Led the CS Scholars section, comprised primarily of students with little to no background in computer science and from typically underrepresented demographic groups.
\end{itemize}
\pagenumbering{gobble}
\newpage

\noindent
\textbf{{\Large Organizations}}\\[-2mm]
\HRule\\
\textbf{Tau Beta Pi, Engineering Honor Society} \\
\textbf{Information Technology Chair}
\hfill 01/2015 -- 05/2016
\begin{itemize}
\vspace{-2.5mm}
\item Oversaw a 4-member team in Django full-stack development, maintaining a strict code review system and requiring comprehensive unit-testing and style adherence.
\vspace{-2.5mm}
\item Led the development and deployment of the Tau Beta Pi Alumni Database, connecting Tau Beta Pi members to alumni mentors throughout industry and academia.
\vspace{-2.5mm}
\item Coordinated Tau Beta Pi website hackathons, where participants work in teams on novel website features; notable projects include search autocompletion, Alumni Database prototype, and tools for website analytics.
\end{itemize}
\vspace{-2mm}

\noindent
\textbf{Professional Development Officer}
\hfill 09/2014 -- 12/2014
\begin{itemize}
\vspace{-2.5mm}
\item Held mock interviews and critiqued resumes for Tau Beta Pi members and the broader engineering community.
\vspace{-6.5mm}
\item Coordinated professional skills events, featuring workshops on Analytical Problem Solving, People Skills and Cultural Awareness.
\end{itemize}
\vspace{-2mm}

\noindent
\textbf{Member} (Tau Beta Pi accepts the top 25\% seniors in the College of Engineering)
\hfill 05/2014 - \textit{present}
\vspace{3mm}
% \begin{itemize}
% \vspace{-1.8mm}
% \item Tau Beta Pi accepts the top 25\% seniors in the College of Engineering
% \end{itemize}
% \noindent
% \textbf{Alpha Chi Sigma, Professional Chemistry Fraternity} \\
% \textbf{Webmaster}
% \hfill 09/2014 -- 12/2014
% \begin{itemize}
% \vspace{-1.8mm}
% \item 
% \end{itemize}

% \noindent
% \textbf{\Large Projects}\\[-2mm]
% \HRule\\
% \textbf{Automatic Panorama Generator} 
% \begin{itemize}
% \vspace{-1.8mm}
% \item Given source images, runs adaptive feature point detection and RANSAC outlier elimination, then learns a homography transformation matrix via least squares
% \end{itemize}
% \vspace{-1.8mm}
% \noindent
% \textbf{Convolutional Neural Networks for Image Compression Artifact Removal}
% \begin{itemize}
% \vspace{-1.8mm}
% \item Trained a deep convolutional neural network to remove JPEG compression artifacts
% \vspace{-2.5mm}
% \item Optimized training using Xavier weight initialization, dropout and fine tuning in the final layer
% \end{itemize}

\noindent
\textbf{\Large Technical Skills}\\[-2mm]
\HRule\\
\textbf{Programming Languages:} Python, Java, C, C++, MATLAB, SQL, Rust, HTML/CSS/JS, \LaTeX \\
\textbf{Software/Frameworks:} Caffe, Tensorflow, Django, Apache Spark, Hadoop MapReduce, scikit-learn, scikit-image

\pagenumbering{gobble}
%\end{resume}
\end{document}
